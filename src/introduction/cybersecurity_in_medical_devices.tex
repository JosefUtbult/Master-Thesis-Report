\section{Cybersecurity in medical devices}

Medical devices are becoming more complex, and more interconnected, with the
rise of \acrfull{iot} capable medical devices. With this comes the risk of
cybersecurity problems for these devices. Faults in software adds attack
surfaces where a device might be compromised by an external actor.

One example of when lack of cybersecurity has affected medical devices is the
case of St. Jude Medical's Implantable Cardiac Devices.

St. Jude Medical's Implantable Cardiac Devices provides different
assistance for people with heart conditions.
These devices include pacemakers,
defibrillators, and resynchronisation devices. 
These are connected to a \textit{Merlin@home Transmitter} via
\acrshort{rf}, which in turn relays the data to the  
\textit{Merlin.net Patient Network} over WiFi, cellular network or
landline.

In January 2017, the \acrfull{fda} issued a corrections request for
the firmware in these devices, as they had reviewed information
concerning potential cybersecurity vulnerabilities associated with St.
Jude Medical's \textit{Merlin@home Transmitter}. These possible
exploits could in theory allow for an unauthorized user to remotely
access a patients \acrshort{rf}-enabled implant, and modify the
programming of these devices, which in turn could result in the rapid
battery depletion. \cite{fda:stJudesMedicalDevices}.

To note is that these request had been issued to the company in
advance, and that they had created a software patch which was
validated to solve these issues by the \acrshort{fda}. This patch was
deployed remotely to all devices connected to the
\textit{Merlin.net Patient Network}. There have been no reports of
patient harm related to these cybersecurity vulnerabilities.

This incident, even though nobody was harmed due to it, shows the
importance of software validation in embedded systems for medical
devices.


