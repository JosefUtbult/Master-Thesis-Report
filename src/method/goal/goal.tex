\subsection{Goals}

The goals for this master thesis is to have made the following.

\subsubsection{Design a PCB}

The project aims to create a platform for audio devices in Rust. For this we will need some hardware 
that can be tested and measured on.

The PCB for this will therefore have to contain the following in order to meet the goals.

\begin{itemize}
	\item A codec for converting analog signals from a microphone or sensor to digital samples for
	processing. 
	It will also need to be able to convert the digital samples back to analog signals for playback 
	on a speaker or headphones.
	This codec needs to be able to process four analog input signals, and four analog output signals.
	%
	\item A microprocessor which has support for the protocol used by the codec, and support for high 
	speed USB.
	%
	\item A USB 2.0 port, or a USB type-C port using the USB 2.0 protocol.
	%
	\item A display for displaying metrics such as signal amplitude, current volume and gain.
	%
	\item A potentiometer for each analog input. These should be connected to the microprocessor for 
	controlling the gain and/or volume of the specific audio input.
	%
	\item A pin header which exposes GPIO pins to the microprocessor, used for easy implementation of 
	future parts for proof of concept designs \todo{Reformulate}.
	%
	\item A pin header which exposes the different wires between the microprocessor and the codec, 
	for debugging purposes.
\end{itemize}

The goal with the PCB design is to start by creating a board without a microprocessor, which functions 
as a HAT \todo{Define HAT} for an external development board.
This design is then extended to incorporate the microprocessor, which will let it function in a 
stand-alone mode.

\subsubsection{Software layer}

The software layer that this project aims to create should function as a platform on which future 
software can be implemented upon.
This software layer is the main object of this thesis, and it needs to incorporate the following.

\begin{itemize}
	\item Communication with the codec for initialization and streaming of data. This software needs to 
	be extendable for allowing exchanging of the codec.
	%
	\item Exposing of the sample input and output data in some sort of queue, where an implementation 
	that uses the software layer should 
	be able to read the samples, process it and write samples back.
	%
	\item The option to include a USB audio class layer, where an implementation should be able to read 
	and write samples to a USB host. 
	This should be made compatible with the queue system of the codec, so that an implementation can 
	read samples from the codec, 
	process them and send them to the USB host (and vice versa).
	%
	\item Unit tests for all the functionality of the software layer to prove software stability and 
	robustness.
\end{itemize}

\subsubsection{Documentation}

This software layer should be documented on how each function in it works, how the overall structure of 
the software layer works, and examples on how to implement it.

\subsubsection{Signal testing}

Tests should be run on the PCB and the running software. The following needs to be measured.

\begin{itemize}
	\item Signal to noise ratio \todo{Add description of signal to noise ratio}.
\end{itemize}
\todo{Add appropriate signal tests}
