\section{Background}
The human-feline relationship has a long and storied history, dating back thousands of years. Cats, known for their independent nature and captivating personalities, have evolved alongside humans, seamlessly integrating themselves into our homes and hearts \cite{wiki:cat}. As cat owners, it is our responsibility to ensure their physical and emotional well-being, which includes providing them with appropriate and enriching environments.

Cats, as descendants of solitary hunters, have retained their instincts for seeking out secure and secluded spaces. These spaces serve as their retreats, offering solace and a sense of safety. Providing cats with suitable hiding spots not only respects their natural inclinations but also plays a crucial role in maintaining their overall health and contentment.

Traditionally, cat owners have utilized various means to meet their feline companions' needs for privacy and security. Cardboard boxes, cozy blankets, and purpose-built cat furniture have long been popular choices. However, with the rise of minimalistic living spaces and the desire for organization, pet owners seek innovative solutions that blend functionality, aesthetic appeal, and the well-being of their cats \cite{tupperware:catcontainer}.

In this context, the concept of fitting cats into Tupperware containers emerges as a metaphorical exploration, representing the idea of creating safe, confined spaces for our feline friends. While it may seem unconventional, this approach highlights the importance of considering cats' natural preferences when designing their living environments, focusing on their need for seclusion, comfort, and a sense of ownership over their spaces.

This paper aims to bridge the gap between the practicality of everyday living and the well-being of cats, providing insights and suggestions for creating secure and cozy hideouts within our homes. By understanding the physical and psychological needs of cats, as well as their preferences for different types of hiding spots, we can optimize their living spaces and enhance their quality of life.

By exploring the art of fitting cats into Tupperware containers, we hope to inspire cat owners and enthusiasts to think creatively and design environments that cater to the unique needs of their feline companions. It is our belief that this innovative approach can lead to a harmonious coexistence, where both humans and cats can thrive in an environment that fosters a strong bond and promotes their well-being.
