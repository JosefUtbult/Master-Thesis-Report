%-------------------------------------------------------------------
% Build definitions
%-------------------------------------------------------------------

<<<<<<< HEAD
% Uncomment this line if you want to compile the document in dark mode
\def\DARKMODE{}

% Uncomment this line to use a high resolution version of the LTU logo.
% Takes longer to compile, so it should only be used for the final 
% compilation, and not during development
%\def\HIGHRES{}
=======
% Comment this out to get a light theme document
%\def\DARKMODE{}

% Comment this out to get the LTU logo in a high resolution. Use the low resolution when writing to minimize compile time, and the higher when the document is published
<<<<<<< HEAD
\def\HIGHRES{}
>>>>>>> 5541739 (Add TODO functionality)
=======
%\def\HIGHRES{}
>>>>>>> 2ebe21c (Include TODO functionality)

% Comment this line out if you want to remove all TODO tags, and the list of TODOs
\def\TODOLIST{}

%-------------------------------------------------------------------
% Document class and package definitions
%-------------------------------------------------------------------

% Apply the document class from SRTs thesis template
% The original template can be found here: https://www.overleaf.com/latex/templates/ltu-srt-thesis-template/jnqrnxdqxrvx
\documentclass[12pt,a4paper,openright,final,twoside]{LTU-Report-Theme/cseethesis}

% Use the theme package defined in LTU-Report-Theme, to apply colors
\usepackage{LTU-Report-Theme/theme}

\begin{document}

% Apply the bibliography file `refs.bib`
\defaultbibliography{refs}

% Apply default library style with the IEEEtran template
\bibliographystyle{IEEEtran}

% Add img/ as a graphics path if you want
%\graphicspath{{img/}}

%-------------------------------------------------------------------
% Define title, author, etc.
%-------------------------------------------------------------------

% TODO: Change title
\def\thesistitle{Building an audio platform for embedded devices in Rust}

% TODO: Change author
\def\theauthor{Josef Utbult}
\def\theaddress{Dept.\ of Computer Science, Electrical and Space Engineering\\
Luleå University of Technology\\ Luleå, Sweden}

% TODO: Change supervisors
\def\supervisors{Per Lindgren}

% TODO: Change to `Supervisors:` if you have more than one
\def\supervisorstring{Supervisor:} 

% TODO: Change dedication
\def\dedication{}

% TODO: Read abstract and preface from separate files.
% Make sure these exist. See example files.
\def\theabstract{\todo{Begin with abstract}
}
\def\thepreface{\input{preface/preface.tex}}

% Create preamble pages, with the LTU logo
\createpreamble
  {\thesistitle}
  {\theauthor}
  {\theaddress}
  {\supervisors}
  {\dedication}
  {\theabstract}
  {\thepreface}

\startchapters
\begin{bibunit}[IEEEtran]
	%------------------------ Start chapter 1 --------------------------
	% The \makechapter command takes three arguments
	%  1) An abbreviated version of the chapter name,
	%     to be used as page header
	%  2) String to be added to the table of contents
	%  3) The chapter name, possibly split in to lines,
	%     as in Chapter 2 below.
	%
	%  The different arguments can have different line breaks.
	%
	% The actual contents of the chapter is included by removing the
	% comment from the \input line below. Make sure the files 
	% "introduction/introduction.tex" and "background/background.tex"
	% exists.
	%-------------------------------------------------------------------

	\def\myquote{``This report, by its very length, defends itself against the risk of being read."\\[.5\baselineskip] Winston Churchill}
\makechapter[\myquote]{Introduction}{Introduction}{Introduction: Why do we put cats in containers?\label{introduction}}
\todo{Add an Introduction}

	\makechapter{Background}{Background}{Background\label{background}}
\todo{Add Background}

	\section{Theory}

	\makechapter{Method}{Method}{Method\label{method}}
\todo{Add requirements to method}
\todo{Add task list and sprint plan in method}

\section{Goals}

The goals for this master thesis is to have made the following.

\subsection{Design a PCB}

The project aims to create a platform for audio devices in Rust. For this we will need some hardware 
that can be tested and measured on.

The PCB for this will therefore have to contain the following in order to meet the goals.

\begin{itemize}
	\item A codec for converting analog signals from a microphone or sensor to digital samples for
	processing. 
	It will also need to be able to convert the digital samples back to analog signals for playback 
	on a speaker or headphones.
	This codec needs to be able to process four analog input signals, and four analog output signals.
	%
	\item A microprocessor which has support for the protocol used by the codec, and support for high 
	speed USB.
	%
	\item A USB 2.0 port, or a USB type-C port using the USB 2.0 protocol.
	%
	\item A display for displaying metrics such as signal amplitude, current volume and gain.
	%
	\item A potentiometer for each analog input. These should be connected to the microprocessor for 
	controlling the gain and/or volume of the specific audio input.
	%
	\item A pin header which exposes GPIO pins to the microprocessor, used for easy implementation of 
	future parts for proof of concept designs \todo{Reformulate}.
	%
	\item A pin header which exposes the different wires between the microprocessor and the codec, 
	for debugging purposes.
\end{itemize}

The goal with the PCB design is to start by creating a board without a microprocessor, which functions 
as a HAT \todo{Define HAT} for an external development board.
This design is then extended to incorporate the microprocessor, which will let it function in a 
stand-alone mode.

\subsection{Software layer}

The software layer that this project aims to create should function as a platform on which future 
software can be implemented upon.
This software layer is the main object of this thesis, and it needs to incorporate the following.

\begin{itemize}
	\item Communication with the codec for initialization and streaming of data. This software needs to 
	be extendable for allowing exchanging of the codec.
	%
	\item Exposing of the sample input and output data in some sort of queue, where an implementation 
	that uses the software layer should 
	be able to read the samples, process it and write samples back.
	%
	\item The option to include a USB audio class layer, where an implementation should be able to read 
	and write samples to a USB host. 
	This should be made compatible with the queue system of the codec, so that an implementation can 
	read samples from the codec, 
	process them and send them to the USB host (and vice versa).
	%
	\item Unit tests for all the functionality of the software layer to prove software stability and 
	robustness.
\end{itemize}

\subsection{Documentation}

This software layer should be documented on how each function in it works, how the overall structure of 
the software layer works, and examples on how to implement it.

\subsection{Signal testing}

Tests should be run on the PCB and the running software. The following needs to be measured.

\begin{itemize}
	\item Signal to noise ratio \todo{Add description of signal to noise ratio}.
\end{itemize}
\todo{Add appropriate signal tests}

\subsection{Plan}

\todo{Should this be in the report?}
To be able to complete this project, the following parts needs to be done.

\begin{itemize}
	\item Studying of relevant topics such as the \textit{USB protocol}, the \textit{USB Audio Class}, 
	implementation the \textit{I2S protocol} and the \textit{STM32 Serial Audio Interface (SAI)} 
	functionality.
	%
	\item Selecting a high quality codec that can handle four input and four output channels, all 
	running at a sample rate of 48000 Hz.
	%
	\item Selecting a microprocessor that can interface with the chosen codec, and also has support for
	high speed USB. A development board for this specific microprocessor also needs to be available.
	%
	\item Designing and building of a PCB that can interface with a development board, containing a codec, 
	a display, potentiometers and a pin header for debugging the wires between the development board and 
	the codec.
	%
	\item Study the existing software modules for I2S communication and USB Audio classes. This is to make
	sure that they contain all the necessary components needed for this project \todo{Such as?}. 
	If these parts aren't present, they will need to be implemented.
	%
	\item Creating an architecture of the software layer, containing how each part of it should work and 
	the flow of information.
	%
	\item Build a software layer that follows the described architecture.
	%
	\item Create unit tests for parts of the software layer that can be feasible tested.
	%
	\item Write documentation for the software layer.
	%
	\item Create and build a final PCB that can be used without the development board.
	%
	\item Test the audio signals for the board.
\end{itemize}

A time plan for the completion of the master thesis can be found in Table \ref{table:timeplan}.
This project started at 6 July 2023, and is set to be finished at 18 October 2023.

The time plan is sectioned of in three sprints. The work that should be done in these sprints will be 
selected at the start of each sprint.

\begin{table}[]
	\caption{Time plan for the completion of the master thesis}
	\todo{I assume this shouldn't be in the final report either}
	\label{table:timeplan}
	\begin{tabular}{lll}
		\textbf{Week} & \textbf{Date} & \textbf{Part} \\
		V23 & 05 jun & Planning \\
		V24 & 12 jun & Rapport structuring + Pre study \\
		V25 & 19 jun & Pre study \\
		V26 & 26 jun & Pre study + selection of codec and devboard \\
		V27 & 03 jul & Sprint 1 \\
		V28 & 10 jun & Sprint 1 \\
		V29 & 17 jun & Sprint 1 \\
		V30 & 24 jun & Sprint 1 + Report writing and evaluation \\
		V31 & 31 jun & Sprint 2 \\
		V32 & 07 aug & Sprint 2 \\
		V33 & 14 aug & Sprint 2 + Report writing and evaluation \\
		V34 & 21 aug & Sprint 3 \\
		V35 & 28 aug & Sprint 3 \\
		V36 & 04 sep & Sprint 3 + Report writing and evaluation \\
		V37 & 11 sep & Final touches on development \\
		V38 & 18 sep & Report writing \\
		V39 & 25 sep & Report writing \\
		V40 & 02 okt & Report writing (First draft) \\
		V41 & 09 okt & Report writing
	\end{tabular}
\end{table}


	\section{Result}
\todo{Add results}

	\makechapter{Discussion}{Discussion}{Discussion\label{discussion}}
\todo{Add Discussion}

	\section{Conclusion}
\todo{Add Conclusion}


	% Add the bibliography at the end
	\makebib
\end{bibunit}
\end{document}

